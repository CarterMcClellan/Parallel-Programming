\documentclass[a4paper, draft]{article}

% Common header definitions and settings for use in the other documentation.


% Packages
\usepackage[T1]{fontenc}    % Font Encoding
\usepackage{utopia}         % Font
%\usepackage{courier}        % Monospaced font % This messes up the code listings, as it is wider than the default.
\usepackage{titling}        % Make modifications to \maketitle
\usepackage{fancybox}       % Boxes around terminal listings.
\usepackage{fancyvrb}       % Code listings, terminal sessions, etc.
\usepackage{color}          % Define and use colours
\usepackage{tocloft}        % TOC modification
\usepackage{datetime}       % Date formatting
\usepackage{float}          % Define new float types (code)
\usepackage{caption}        % Modify caption appearance
\usepackage{todonotes}      % Add todo notes in the margin
\usepackage{framed}         % shaded environment provides background colours
\usepackage{fourier-orns}   % Ornaments around title
\usepackage{siunitx}        % Align on decimal point column type
\usepackage{ifdraft}        % Test for draft mode
\usepackage{tikz}           % Drawing Trees
\usepackage{pdflscape}      % provides landscape environment for pdflatex.
\usepackage{ifthen}         % Provides conditional tests
\usepackage{hyperref}       % Links, and nameref command




% Define some commannds for displaying the names of different things
\newcommand{\filename}[1]{\texttt{#1}}
\newcommand{\command}[1]{\texttt{#1}}
\newcommand{\var}[1]{$#1$}
\newcommand{\mathvar}[1]{#1}
\newcommand{\codefragment}[1]{\texttt{#1}}
\newcommand{\confsnippet}[1]{\texttt{#1}}

\newcommand{\class}[1]{\texttt{#1}}
\newcommand{\func}[1]{\texttt{#1}}



% Set up the appearance of todo notes
\newcommand{\todonote}[1]{\todo[backgroundcolor=white, linecolor=black, 
                                size=\small, noline]{#1}}



% Environment for showing bits of code, terminal listings, and so on.
\definecolor{box-grey}{gray}{0.4}
\newcommand{\codelabel}[1]{\textcolor{black}{#1}}
\DefineVerbatimEnvironment
    {Code}
    {Verbatim}
    {frame=single, fontsize=\scriptsize, framesep=3mm, formatcom=\vspace{7pt}, 
      xleftmargin=5mm, xrightmargin=5mm, rulecolor=\color{box-grey},
      labelposition=topline, numbersep=8pt, baselinestretch=1.2,
      numbers=left, samepage=true}
% N.B. can use option label=\codelabel{??} as needed.



% Wrapper for the above for teminal listings
% This is a pretty nasty hack, but I'm really tired. ;)
\definecolor{shadecolor}{gray}{0.95}
\setlength{\FrameSep}{-5mm}
\newenvironment{Term}
    {\begin{center}\begin{Sbox}\begin{minipage}{\linewidth-17.4mm}}
    {\end{minipage}\end{Sbox}{\setlength{\fboxsep}{3mm}\vspace*{8pt}\fcolorbox{box-grey}{shadecolor}{\TheSbox}\vspace*{6pt}}\end{center}}

\DefineVerbatimEnvironment
    {CodeTerm}
    {Verbatim}
    {frame=none, fontsize=\scriptsize, framesep=3mm, formatcom={}, 
      xleftmargin=0mm, xrightmargin=0mm, rulecolor=\color{box-grey},
      labelposition=topline, numbersep=8pt, baselinestretch=1.2,
      numbers=left, samepage=true,
      numbers=none}

% Usage:
%\begin{Code}
%\begin{CodeTerm}
%Hello, World.
%\end{CodeTerm}
%\end{Term}


% Rename the abstract
\renewcommand{\abstractname}{Introduction}



% Remove section numbering
\setcounter{secnumdepth}{0}



% Remove bold and add dots to sections in the TOC
\renewcommand\cftsecfont{\normalfont}
\renewcommand\cftsecpagefont{\normalfont}
\renewcommand{\cftsecleader}{\cftdotfill{\cftsecdotsep}}
\renewcommand\cftsecdotsep{\cftdot}
\renewcommand\cftsubsecdotsep{\cftdot}
% Show only sections in the TOC
\setcounter{tocdepth}{1}


% Configure the \maketitle command a little
\setlength{\droptitle}{-30pt}
\predate{\begin{center}\small}
\postdate{\par\end{center}}



% Set up the desired date format (20^th July 2011)
% N.B. this is very similar to the default with the 'nodayofweek' option.
\newdateformat{mydate}{\ordinaldate{\THEDAY} \monthname[\THEMONTH] \THEYEAR}
\mydate



% New float type for code listings
\floatstyle{plain}
\floatname{listing}{Listing}
\newfloat{listing}{p}{lol}



% Set up appearance of captions
\captionsetup{margin=30pt,font=small,labelfont=bf}



% Remove ugly boxes etc around hyperlinks.
\hypersetup{pdfborder = {0 0 0}}




% Tree Drawing
% {A, B, {C, D}, {E, F}}
\newcommand{\treeDrawABCDEF}{%
    \[\begin{tikzpicture}
        \node{$\{A, B\}$}
            child{node {$\{C, D\}$}}
            child{node {$\{E, F\}$}}
        ;
    \end{tikzpicture}\]
}




% Add a command to produce nice looking tildes: \urltilde
% http://stackoverflow.com/questions/718479/what-is-the-best-way-to-produce-a-tilde-in-latex-for-a-website/3900556#3900556
\catcode`~=11 % make LaTeX treat tilde (~) like a normal character
\newcommand{\urltilde}{\kern -.15em\lower .7ex\hbox{~}\kern .04em}
\catcode`~=13 % revert back to treating tilde (~) as an active character




% Allow document title to be set from main file.
\newcommand{\thedocumenttitle}{}
\newcommand{\documenttitle}[1]{%
    \renewcommand{\thedocumenttitle}{#1}
}



% Title Info
\title{Flamingo Auto-Tuning\\\decofourleft~~~\thedocumenttitle~~~\decofourright}
\author{Ben Spencer\\\normalsize\href{mailto:ben@mistymountain.co.uk}{ben@mistymountain.co.uk}}
\date{Last updated: \today}




\documenttitle{User's Guide}


\begin{document}

\maketitle

\begin{abstract}
This guide describes the features and use of my automatic parameter tuning 
system. The system can be used to find optimal parameter values in a 
parameterised program. It has been designed to be very general-purpose and 
makes few assumptions about how it should be used.

If you have any comments or questions about the tuner or this guide then
please feel free to get in touch.
\end{abstract}


\tableofcontents

\newpage


\section{Background}
\label{sec:intro}
The auto-tuner is used to find the optimal settings of program parameters. 
These would often be constants within the program which are used to control 
some aspect of its operation. If it is unclear exactly which combination of 
settings to use in a particular situation, the tuner can find them. 

If some 
parameters are independent of each other, this independence can be exploited 
to speed up the tuning process. Usually, we want to optimise the running time 
of a program, so this is the default. However, it is possible to provide a 
custom `figure of merit' which is used to rank the different tests.







\section{Tutorial}
\label{sec:tutorial}
This guide will go through the features of the tuner in detail, and can be 
used as a reference. A tutorial is available, which should probably be read 
first. It will lead you through the process of preparing and tuning an 
example program, explaining each step. This tutorial is 
\filename{doc/tutorial.pdf}, distributed with the tuner.







\section{The Configuration File}
\label{sec:conf-file}
The tuner uses a configuration file to determine what to optimise, and how 
tests can be run. This file is used to set up and run the optimisation.

The file must contain five sections, each beginning with a 
\confsnippet{[section\_name]} header. Within each section, options are set 
using the syntax \confsnippet{opt = value}. Lines that begin with 
\confsnippet{\#} or \confsnippet{;} are treated as comments and ignored.

A template configuration file (\filename{examples/template.conf}) is provided, 
containing some explanation of each option. The example programs also each come 
with a sample configuration file for tuning them.

\textbf{All paths and commands used must be relative to the configuration 
file.}

\subsection{The \confsnippet{[variables]} Section}
This section defines the program parameters which are to be optimised, and 
any independence between them. There is only a single option:

\begin{description}
    \item[\confsnippet{variables}] (Required) \\
        This option gives a list of the program parameters which should be 
        optimised. There are two possible formats:
        \begin{itemize}
            \item A simple comma separated list of variable names, 
                for example: \\
                \confsnippet{variables~=~FOO,~BAR,~BAZ}

            \item A list describing the independence between variables, 
                as described in the \emph{\nameref{sec:var-indep}} section. 
                For example: \\
                \confsnippet{variables~=~\{\{FOO\},~\{BAR,~BAZ\}\}} \\
        \end{itemize}
\end{description}



\subsection{The \confsnippet{[values]} Section}
This section gives the possible values that each variable above can take. They 
are specified as a comma separated list for each variable.

Each variable from the \confsnippet{[variables]} section must have an entry 
here. Any entries here which are not mentioned in the variable list will 
be ignored.

All values are interpreted as text strings, which are used without conversion, 
for example as compiler flags or the program's arguments.




\subsection{The \confsnippet{[testing]} Section}
This section defines how to compile, if needed, and run the tests. Tests can 
also be removed once their testing is complete, if needed.

All the options in this section specify commands which will be executed by the 
tuner. The value of any variable from the \confsnippet{[variables]} section 
can be substituted into the command which is run, using the syntax 
\confsnippet{\%VAR\_NAME\%}.

For example, if there is a variable \var{FOO} being optimised, then the 
\confsnippet{test} command might be: \confsnippet{./myTestProgram~-op~\%FOO\%}. 
If for a particular test the variable \var{FOO} takes the value \var{3}, then 
to get a score for that test, the command \confsnippet{./myTestProgram~-op~3} 
is executed and timed.

There is also one special substitution: \texttt{\%\%ID\%\%}. 
This provides a unique id for each test (which is a counter increasing from 1).


\begin{description}

    \item[\texttt{compiler}] (Optional) \\
        If specified, this command is executed before testing begins. 
        It can be used to compile a test if a parameter is benig changed at 
        compile time (for example a compiler flag or a \codefragment{\#define} 
        statement being overridden by the compiler).

    \item[\texttt{test}] (Required) \\
        To perform a test, this command will be executed and timed by the 
        tuning system. The running time of the command is taken as the score 
        for each test. If a custom figure-of-merit is being used (chosen with 
        the \confsnippet{optimal} option below), then the running time is not 
        measured and instead the score is read from the final line of output 
        from this command.

    \item[\texttt{cleanup}] (Optional) \\
        If specified, this command is executed once the test is complete. 
        It can be used to remove any executables which were compiled.

\end{description}




\subsection{The \confsnippet{[scoring]} Section}
This section specifies how tests are scored against each other to choose 
which is the best.

\begin{description}

    \item[\confsnippet{repeat}] (Optional, defaults to \confsnippet{1, min}) \\
        Gives the number of times each test should be repeated. When 
        this option is set, the compilation and cleanup (if present) are 
        still only run once, only the actual test is repeated. The variance, 
        standard deviation and coefficient of variation will be displayed at 
        the end of each test.
        
        When a test is repeated, the repeated test scores must be combined 
        into one overall score for the test. This can be one of the following: 
        \confsnippet{min}, \confsnippet{max}, \confsnippet{med} or 
        \confsnippet{avg}. 
        
        The number of repetitions and the aggregation function are specified 
        as a comma separated pair. If only the number is given, 
        \confsnippet{min} is used as the default aggregate.

    \item[\confsnippet{optimal}] (Optional, defaults to \texttt{min\_time}) \\
        Specifies whether to take the minimum or maximum score as being the 
        best. The settings \confsnippet{min\_time} and \confsnippet{max\_time} 
        use the running time of the \confsnippet{test} command as the score 
        for a test. The settings \confsnippet{min} and \confsnippet{max} will 
        use a custom figure-of-merit, which is read from the final line of 
        output. This is described in the \emph{\nameref{sec:custom-fom}} 
        section.

\end{description}


\subsection{The \confsnippet{[output]} Section}
Finally, a log can be saved, detailing the testing process:

\begin{description}

    \item[\confsnippet{log}] (Optional) \\
        Specifies the name of a \filename{.csv} file which will be generated. 
        The file will contain details of all the tests performed, which 
        parameter values were used for each and what the individual and 
        overall scores were. This can be used for more detailed analysis 
        after tuning. 
        
        \textbf{Warning:} If this file already exists, it will be overwritten.
    
    \item[\confsnippet{script}] (Optional) \\
        Specifies the name of a file which will be used to log a `script' of 
        the tuner's work. This includes which tests are being run and any 
        output from the compilation and testing.
        
        When this option is used, only a summary is shown on screen.
        
        \textbf{Warning:} If this file already exists, it will be overwritten.
    
\end{description}








\clearpage


\section{The Example Programs}
There are some example programs to tune in the \filename{examples} directory.
These show how a few different parameterised programs can be tuned, the 
changes required to the programs and makefiles, and the configuration files 
required.

\begin{description}
    
    \item[\filename{examples/hello/}] \hfill \\
        A trivial test case, which demonstrates how different parts of the 
        system are connected. It compiles a `hello world' program, 
        written in C.
        
        Two parameters, named \var{FOO} and \var{BAR}, are completely 
        ignored and one, \var{OPTLEVEL}, controls the compiler 
        optimisation level flag.
        
        The running time of the program is used as a figure of merit, aiming 
        to find the minimum (fastest). There are also example settings in the 
        configuration file to optimise the file size of the generated 
        executable.
        
    \item[\filename{examples/laplace3d/}] \hfill \\
        Uses compile-time parameters to tune the block size used by a CUDA GPU 
        program.
    
    \item[\filename{examples/looping/}] \hfill \\
        This is another fairly trivial C program, which runs a loop as 
        determined by the parameters \var{XLOOP} and \var{YLOOP}.
        
        There are settings in the configuration file to measure the running 
        time as measured by the tuner, or by using a custom script, 
        \filename{loop\_test.sh}, which uses the \command{time} utility.
        
    \item[\filename{examples/maths/}] \hfill \\
        A very simple test case to demonstrate the use of a custom figure of 
        merit. The parameters \var{X}, \var{Y} and \var{Z} are simply summed 
        using the \command{expr} utility, with no compilation being performed.
        
    \item[\filename{examples/matlab/}] \hfill \\
        Demonstrates the use of run-time parameters to find determine the 
        optimum level of `strip-mining' vectorisation in a MATLAB program.
    
    \item[\filename{examples/matrix/}] \hfill \\
        Blocked matrix multiplication test case. This is the example used in 
        the tutorial to demonstrate the changes required to the program. The 
        original (not tunable) version is given in \filename{original/}, 
        the modified version after completing the tutorial is given in 
        \filename{modified/} and finally, there is a version which 
        checks the blocked version against the naive version in 
        \filename{comparison/}.
        
    
\end{description}




\clearpage


\section{The Utilities}
The tuner comes with several utilities which can be used to analyse or 
visualise the results of the tuning process. They use \filename{.csv} log 
files generated by the tuner. The utilities are all found in the 
\filename{utilities} directory.



\begin{description}

    \item[\filename{utilities/output\_gnuplot.py}] \hfill \\
        This script converts a CSV log file into a gnuplot PLT file. This PLT 
        file can be used with the gnuplot plotting program to produce a 
        detailed graph of the testing process. If required, the PLT file can 
        be modified by hand. The following options can be specified:
        \begin{description}
            \item[\command{-h} \textnormal{or} \command{-{}-help}] \hfill\\
                Outputs some usage information and exits.
            
            \item[\command{-r SCORE} \textnormal{or} \command{-{}-reference=SCORE}] \hfill\\
                Plots a reference score for comparison with the tuner's results.
            
        \end{description}
        
        
        If \filename{mylog.csv} is the tuning log and \filename{myplot.plt} is 
        the plot file to generate (it will be overwritten if it exists):
        \begin{Code}[numbers=none]
./output_gnuplot.py [-h] [-r SCORE] mylog.csv myplot.plt
        \end{Code}


    \item[\filename{utilities/output\_screen.py}] \hfill \\
        This script reads a CSV log file and produces a graph displayed on the 
        screen. This can then be saved if needed. The 'matplotlib' python 
        library is required, which may not be installed by default. 
        The following options can be specified:
        \begin{description}
            \item[\command{-h} \textnormal{or} \command{-{}-help}] \hfill\\
                Outputs some usage information and exits.
            
            \item[\command{-r SCORE} \textnormal{or} \command{-{}-reference=SCORE}] \hfill\\
                Plots a reference score for comparison with the tuner's results.
            
            \item[\command{-s} \textnormal{or} \command{-{}-stddev}] \hfill\\
                Plots the standard deviation of multiple test repetitions.
            
        \end{description}
        
        If \filename{mylog.csv} is the tuning log:
        \begin{Code}[numbers=none]
./output_screen.py [-h] [-r SCORE] [-s] mylog.csv
        \end{Code}


    \item[\filename{utilities/csv\_plot.m}] \hfill \\
        This is a MATLAB program which can be used to display a graph of the 
        testing process.
        
        To use, modify the file as needed and use MATLAB to generate a graph.


\end{description}





\clearpage

\section{Variable Independence}
\label{sec:var-indep}
We say two variables are independent when they can be optimised separately. 
This is, \var{FOO} and \var{BAR} are independent if when we optimise 
\var{FOO} with \var{BAR} held at some fixed value, then that optimal 
value of \var{FOO} will be optimal for \emph{any} setting of \var{BAR} 
(and vice-versa). 

Independent variables are written as separate `sets' of 
variables. The example we just saw would be written as:
\[\{\{FOO\},~\{BAR\}\}\]

More complex independences can also be written. As an example, suppose 
\var{A} and \var{B} control the operation of the entire program (maybe 
they are compiler flags), \var{C} and \var{D} control one aspect of the 
program and \var{E} and \var{F} another aspect. The aspect controlled 
by \var{C} and \var{D} is not related to the aspect controlled by 
\var{E} and \var{F}. This independence would be written like this: 
\[\{A,~B,~\{C,~D\},~\{E,~F\}\}\]
This shows the two sub-lists \var{\{C,~D\}} and \var{\{E,~F\}} are 
independent of each other and that the variables \var{A} and \var{B} 
`dominate' those sub-lists. However, the sub-list \var{\{C,~D\}} shows that 
\var{C} and \var{D} are dependent on each other and must be tuned together. 
Similarly, \var{A} and \var{B} are dependent on each other.

This notation essentially describes a `tree' of 
variables, where higher nodes dominate their subtrees and sibling nodes are 
independent:
\treeDrawABCDEF

Variable lists can be nested in this way as much as needed to describe 
the independences in the parameters being optimised.





\section{Figures of Merit}
\label{sec:custom-fom}
By default (when using \confsnippet{optimal = *\_time}), each test is run and 
its execution time is measured. This timing is used to rank the tests and 
choose the best. Sometimes, it is more useful to be able to choose a custom 
figure of merit. This can be because the part of the program being optimised 
is not the longest running part of the program, or because you wish to measure 
some other property, such as memory or network usage.

When the \confsnippet{optimal} option is set to \confsnippet{min} or 
\confsnippet{max}, the auto-tuner reads the output from the program and 
interprets the last line of output as the figure of merit. 
You are free to make any measurements necessary within the test itself 
(given by the \confsnippet{test} option). The score (a floating point or 
integer number with no units or other text) for the test must be output as 
the final line of the command's output.










\clearpage

\section{Data Movement}
Because arbitrary commands can be used for compilation, execution and cleaning, 
it is possible to use the parameters in any way you could from a command 
prompt. However, it is not always obvious how the parameter values can be 
`passed' through the toolchain to where they are needed. The following list of 
tips may help. Although they are fairly linux/\command{make}/\command{gcc} 
specific, the ideas should still be applicable to any build tools.

%It can be useful to check that the compilation and testing commands are 
%correctly passing the parameters through the build chain and returning any 
%figure-of-meit back to the tuner. This can be checked by only giving 
%one possible value for each parameter in the configuration file, so a tuning 
%run will require exactly one test.

To perform a `dry-run' with only one test to check these settings, 
simply set a single possible value for each parameter.

\begin{description}

\item[Tuner $\Longrightarrow$ Shell Command] \hfill\\
    To pass the value of a parameter to a shell command, simply use the \% 
    substitution. If you are tuning a parameter named \var{FOO} (in the 
    variable list) then \confsnippet{\%FOO\%} will substitute the current 
    value of \var{FOO} being tested into the command to execute.
    
    There is also a special substitution, \confsnippet{\%\%ID\%\%}, which is 
    replaced by a 
    
    This can be used to set compiler flags, program arguments, and so on. For 
    example, \filename{hello.conf} contains the following lines:
    % Do some funky (read: hacked) line numbering on this example...
    {
    \let\theFancyVerbLineOld\theFancyVerbLine
    \renewcommand{\theFancyVerbLine}{%
        \ifthenelse{\value{FancyVerbLine} = 1}{%
            \setcounter{FancyVerbLine}{49} %
            \theFancyVerbLineOld %
        }{}%
        \ifthenelse{\value{FancyVerbLine} = 51}{%
            \setcounter{FancyVerbLine}{56} %
            \theFancyVerbLineOld %
        }{}%
        \ifthenelse{\value{FancyVerbLine} = 58}{%
            \setcounter{FancyVerbLine}{66} %
            \theFancyVerbLineOld %
        }{}%
    }
    \begin{Code}[label=\codelabel{examples/hello/hello.conf},
                    commandchars=\\\{\}]
compile = gcc %OPTLEVEL% -o bin/test_%%ID%% hello.c
\ldots{}
test = ./bin/test_%%ID%% %FOO% %BAR%
\ldots{}
clean = rm ./bin/test_%%ID%%
    \end{Code}
    }
    
    This will cause the following commands to be executed by the tuner:
    \begin{Code}[numbers=none, commandchars=\\\{\}]
gcc -O0 -o bin/test_1 hello.c
./bin/test_1 1 1                                                  (Timed)
./bin/test_1 1 1                                                  (Timed)
./bin/test_1 1 1                                                  (Timed)
rm ./bin/test_1
gcc -O0 -o bin/test_2 hello.c
./bin/test_2 2 1                                                  (Timed)
./bin/test_2 2 1                                                  (Timed)
./bin/test_2 2 1                                                  (Timed)
rm ./bin/test_2
\ldots{}
    \end{Code}
    
    The commands you enter are executed directly, as if typed into a command 
    prompt. You are not limited to only running a compiler, or just your test 
    program. It can be useful to use a makefile for compilation or a shell 
    script for measuring and returning a custom figure of merit.



\item[Tuner $\Longrightarrow$ Makefile] \hfill\\
    Parameters can be passed to the \command{make} command simply by appending 
    \command{NAME=VALE} to the call to \command{make}, which will allow 
    them to be used within a \filename{Makefile}. However, changing the 
    parameters does not trigger a recompilation of the affected files 
    (because there has been no change to the source code), so this must be 
    forced with the \command{-B} option to \command{make}.
    
    For example, \filename{looping.conf} contains the following:\footnote{Note 
    that \var{EXEC\_NAME} is a new variable for the \filename{Makefile} only, 
    it is not being tuned.}
    \begin{Code}[firstnumber=50, stepnumber=2, firstline=2,
                    label=\codelabel{examples/looping/looping.conf}]
    
compile = make -f MakeLoop -B EXEC_NAME=./bin/loop_%%ID%% XLOOP=%XLOOP% \
                            YLOOP=%YLOOP% OPTLEVEL=%OPTLEVEL%
    \end{Code}
    
    These variables can now be used within the \filename{Makefile} as if they 
    were environment variables, with the syntax \command{\$(NAME)}.





\item[Makefile $\Longrightarrow$ Compiler] \hfill\\
    Once the parameters have been passed into the \filename{Makefile} as described 
    above, they can be used in the commands to be executed. For example, the 
    \filename{Makefile} for the `looping' example contains the following, 
    which passes the parameters to the compiler:
    \begin{Code}[firstnumber=4, label=\codelabel{examples/looping/MakeLoop}]
gcc $(OPTLEVEL) -o $(EXEC_NAME) loop.c -D XLOOP=$(XLOOP) -D YLOOP=$(YLOOP)
    \end{Code}






\item[Compiler $\Longrightarrow$ Program Code] \hfill\\
    As shown above, the `looping' \filename{Makefile} uses the \command{-D} 
    option to \command{gcc} to set the parameter values for \var{XLOOP} and 
    \var{YLOOP}. This flag sets them as if they had been set with a 
    \codefragment{\#define} statement in the program itself, so they can be 
    used as constants within the program source code.
    



\item[Tuner $\Longrightarrow$ Environment Variable] \hfill\\
    Some programs may be controlled by a certain environment variable, which 
    you want to tune. For example, programs using OpenMP for parallelism 
    can have the number of parallel threads controlled with the environment 
    variable \command{OMP\_NUM\_THREADS}. To tune this variable, the 
    \command{env} command can be used to run the program to be tested in 
    an environment where \command{OMP\_NUM\_THREADS} has the value we want:
    \begin{Code}[numbers=none]
test = env OMP_NUM_THREADS=%OMP_NUM_THREADS% ./program
    \end{Code}
    
    
    
\item[Command Sequencing] \hfill\\
    The commands given in the \confsnippet{compile}, \confsnippet{test} and 
    \confsnippet{clean} options are passed directly to the shell, so features 
    of the shell, such as using \command{;} to run one command after another, 
    can be used. For example, another possible solution to the above problem 
    of setting an environment variable would be to export it, then run the 
    command which depends on it:
    \begin{Code}[numbers=none]
test = export OMP_NUM_THREADS=%OMP_NUM_THREADS%; ./program
    \end{Code}
    
    


\end{description}








\clearpage


\section{Return Codes}
The tuner will check the return code of all the commands which are executed 
(\confsnippet{compile}, \confsnippet{test} and \confsnippet{clean}) to see 
if they failed. Any non-zero return code is considered a failure. This check 
is used to discount tests which either do not compile or do not run with one 
particular setting of the parameters. 

This is part of the tuning process, 
as it discounts tests which will not run in your environment which may still 
run elsewhere. For example, if the number of parallel threads is being tuned, 
valid possible values on one machine may fail to compile on machines where 
there are more threads than processors. 

If this causes problems, for example 
if your program routinely returns non-zero error codes, then the 
\confsnippet{test} command could be run from a `wrapper' shell script which 
always succeeds, or the shell's sequencing operator (\command{;}) could be 
used to set the return code afterwards, the following will always return 
with code 0 (success), whatever the behaviour of \command{./program}:
\begin{Code}[numbers=none]
test = ./program; true
\end{Code}

If you do this, the system will not be able to detect failed tests, so they 
will still be timed (and will often fail quickly, leading to very low running 
times), so be careful of erroneous results.






\section{Running the Tuner}
The main script which runs the tuner is \filename{tuner/tune.py}. In the 
top-level directory, \command{autotune} is a link to this and should be used 
to run the tuner.

While the tuner is running tests it can be interrupted with the 
\command{Ctrl+C} command. If the \confsnippet{log} option was used in the 
configuration file, then a partial log of the tests completed so far will 
be saved.



\section{Dependencies}
The tuner is written in \textbf{Python}, so you will need to have the Python 
interpreter installed. At least \textbf{version 2.5} is required, and 
Python 3000 is not supported.

So far, the tuner has mostly been tested under linux, but is
designed to work on any platform (the most compelling reason for choosing 
Python was this portability and flexibility). On linux, the 
tuner will be directly executable with \command{./autotune} or 
\command{./tuner/tune.py}, but on windows it will need to be given as an 
argument to the python interpreter: 
\command{python~tuner\textbackslash{}tune.py}.

The utility \command{output\_screen.py} (which plots graphs of the testing 
process on screen) requires the \filename{matplotlib} 
Python module, which may not be installed with a standard python 
distribution. If it is not installed, you will not be able to use this 
utility, but the rest of the system will still work as normal.


\end{document}





